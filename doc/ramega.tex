% generated by GAPDoc2LaTeX from XML source (Frank Luebeck)
\documentclass[a4paper,11pt]{report}

\usepackage[top=37mm,bottom=37mm,left=27mm,right=27mm]{geometry}
\sloppy
\pagestyle{myheadings}
\usepackage{amssymb}
\usepackage[latin1]{inputenc}
\usepackage{makeidx}
\makeindex
\usepackage{color}
\definecolor{FireBrick}{rgb}{0.5812,0.0074,0.0083}
\definecolor{RoyalBlue}{rgb}{0.0236,0.0894,0.6179}
\definecolor{RoyalGreen}{rgb}{0.0236,0.6179,0.0894}
\definecolor{RoyalRed}{rgb}{0.6179,0.0236,0.0894}
\definecolor{LightBlue}{rgb}{0.8544,0.9511,1.0000}
\definecolor{Black}{rgb}{0.0,0.0,0.0}

\definecolor{linkColor}{rgb}{0.0,0.0,0.554}
\definecolor{citeColor}{rgb}{0.0,0.0,0.554}
\definecolor{fileColor}{rgb}{0.0,0.0,0.554}
\definecolor{urlColor}{rgb}{0.0,0.0,0.554}
\definecolor{promptColor}{rgb}{0.0,0.0,0.589}
\definecolor{brkpromptColor}{rgb}{0.589,0.0,0.0}
\definecolor{gapinputColor}{rgb}{0.589,0.0,0.0}
\definecolor{gapoutputColor}{rgb}{0.0,0.0,0.0}

%%  for a long time these were red and blue by default,
%%  now black, but keep variables to overwrite
\definecolor{FuncColor}{rgb}{0.0,0.0,0.0}
%% strange name because of pdflatex bug:
\definecolor{Chapter }{rgb}{0.0,0.0,0.0}
\definecolor{DarkOlive}{rgb}{0.1047,0.2412,0.0064}


\usepackage{fancyvrb}

\usepackage{mathptmx,helvet}
\usepackage[T1]{fontenc}
\usepackage{textcomp}


\usepackage[
            pdftex=true,
            bookmarks=true,        
            a4paper=true,
            pdftitle={Written with GAPDoc},
            pdfcreator={LaTeX with hyperref package / GAPDoc},
            colorlinks=true,
            backref=page,
            breaklinks=true,
            linkcolor=linkColor,
            citecolor=citeColor,
            filecolor=fileColor,
            urlcolor=urlColor,
            pdfpagemode={UseNone}, 
           ]{hyperref}

\newcommand{\maintitlesize}{\fontsize{50}{55}\selectfont}

% write page numbers to a .pnr log file for online help
\newwrite\pagenrlog
\immediate\openout\pagenrlog =\jobname.pnr
\immediate\write\pagenrlog{PAGENRS := [}
\newcommand{\logpage}[1]{\protect\write\pagenrlog{#1, \thepage,}}
%% were never documented, give conflicts with some additional packages

\newcommand{\GAP}{\textsf{GAP}}

%% nicer description environments, allows long labels
\usepackage{enumitem}
\setdescription{style=nextline}

%% depth of toc
\setcounter{tocdepth}{1}





%% command for ColorPrompt style examples
\newcommand{\gapprompt}[1]{\color{promptColor}{\bfseries #1}}
\newcommand{\gapbrkprompt}[1]{\color{brkpromptColor}{\bfseries #1}}
\newcommand{\gapinput}[1]{\color{gapinputColor}{#1}}


\begin{document}

\logpage{[ 0, 0, 0 ]}
\begin{titlepage}
\mbox{}\vfill

\begin{center}{\maintitlesize \textbf{ GAP 4 Package RAMEGA \mbox{}}}\\
\vfill

\hypersetup{pdftitle= GAP 4 Package RAMEGA }
\markright{\scriptsize \mbox{}\hfill  GAP 4 Package RAMEGA  \hfill\mbox{}}
{\Huge \textbf{ RAndom MEthods in Group Algebras \mbox{}}}\\
\vfill

{\Huge  1.0.0 \mbox{}}\\[1cm]
{ 2019 \mbox{}}\\[1cm]
\mbox{}\\[2cm]
{\Large \textbf{ Zsolt Adam Balogh\\
   \mbox{}}}\\
{\Large \textbf{ Vasyl Laver\\
   \mbox{}}}\\
\hypersetup{pdfauthor= Zsolt Adam Balogh\\
   ;  Vasyl Laver\\
   }
\end{center}\vfill

\mbox{}\\
{\mbox{}\\
\small \noindent \textbf{ Zsolt Adam Balogh\\
   }  Email: \href{mailto://baloghzsa@gmail.com} {\texttt{baloghzsa@gmail.com}}\\
  Address: \begin{minipage}[t]{8cm}\noindent
 Department of Mathematical Sciences\\
 UAEU\\
 Al Ain, United Arab Emirates\\
 \end{minipage}
}\\
{\mbox{}\\
\small \noindent \textbf{ Vasyl Laver\\
   }  Email: \href{mailto://vasyl.laver@uzhnu.edu.ua} {\texttt{vasyl.laver@uzhnu.edu.ua}}\\
  Address: \begin{minipage}[t]{8cm}\noindent
 Department of Mathematical Sciences\\
 UAEU\\
 Al Ain, United Arab Emirates\\
 \end{minipage}
}\\
\end{titlepage}

\newpage\setcounter{page}{2}
{\small 
\section*{Copyright}
\logpage{[ 0, 0, 1 ]}
 {\copyright} 2019 by the authors

 This package may be distributed under the terms and conditions of the GNU
Public License Version 2 or higher. \mbox{}}\\[1cm]
\newpage

\def\contentsname{Contents\logpage{[ 0, 0, 2 ]}}

\tableofcontents
\newpage

  
\chapter{\textcolor{Chapter }{Introduction}}\label{intro}
\logpage{[ 1, 0, 0 ]}
\hyperdef{L}{X7DFB63A97E67C0A1}{}
{
  \textsf{RAMEGA} stands for ``RAndom MEthods in Group ALgebras". This package is dedicated to
realization of some standard functions for group rings by the means of the
random methods. This approach allows to compute the needed function in a
reasonable time and with minimal memory use even for the large group rings \cite{Bovdi_Rosa_I}. 
\section{\textcolor{Chapter }{Overview over this manual}}\label{overview}
\logpage{[ 1, 1, 0 ]}
\hyperdef{L}{X786BACDB82918A65}{}
{
  Chapter \ref{random} describes the functions for group algebras which were obtained using the
random methods. }

 
\section{\textcolor{Chapter }{Installation}}\label{instal}
\logpage{[ 1, 2, 0 ]}
\hyperdef{L}{X8360C04082558A12}{}
{
  To get the newest version of this \textsf{GAP} 4 package download the archive file \texttt{thelma-x.x.tar.gz} and unpack it in a directory called ``\texttt{pkg}'', preferably (but not necessarily) in the ``\texttt{pkg}'' subdirectory of your \textsf{GAP} 4 installation. It creates a subdirectory called ``\texttt{RAMEGA}''.

 As \textsf{RAMEGA} has no additional C libraries, there is no need in any additional installation
steps.

 }

 
\section{\textcolor{Chapter }{Feedback}}\label{feedback}
\logpage{[ 1, 3, 0 ]}
\hyperdef{L}{X80D704CC7EBFDF7A}{}
{
  For bug reports, feature requests and suggestions, please use the github issue
tracker/  }

  }

   
\chapter{\textcolor{Chapter }{Random Methods}}\label{random}
\logpage{[ 2, 0, 0 ]}
\hyperdef{L}{X7D7DE68778F94CA9}{}
{
  
\section{\textcolor{Chapter }{Basic Operations}}\label{random_basic}
\logpage{[ 2, 1, 0 ]}
\hyperdef{L}{X82EB5BE77F9F686A}{}
{
  

\subsection{\textcolor{Chapter }{BasicGroup}}
\logpage{[ 2, 1, 1 ]}\nobreak
\hyperdef{L}{X792693D382FF15B8}{}
{\noindent\textcolor{FuncColor}{$\triangleright$\enspace\texttt{BasicGroup({\mdseries\slshape KG})\index{BasicGroup@\texttt{BasicGroup}}
\label{BasicGroup}
}\hfill{\scriptsize (function)}}\\


  For the group ring \texttt{KG} the function \texttt{BasicGroup} returns the basic group of \texttt{KG} as a subgroup of the normalized group of units. 
\begin{Verbatim}[commandchars=!@|,fontsize=\small,frame=single,label=Example]
  
  !gapprompt@gap>| !gapinput@G:=CyclicGroup(IsFpGroup,4);
|
  <fp group of size 4 on the generators [ a ]>
  !gapprompt@gap>| !gapinput@Elements(G);
|
  [ <identity ...>, a, a^2, a^3 ]
  !gapprompt@gap>| !gapinput@KG:=GroupRing(GF(2),G);
|
  <algebra-with-one over GF(2), with 1 generators>
  !gapprompt@gap>| !gapinput@BasicGroup(KG);
|
  <group with 4 generators>
  !gapprompt@gap>| !gapinput@Elements(last);
|
  [ (Z(2)^0)*<identity ...>, (Z(2)^0)*a, (Z(2)^0)*a^2, (Z(2)^0)*a^3 ]
  
\end{Verbatim}
 }

 

\subsection{\textcolor{Chapter }{IsLienEngel}}
\logpage{[ 2, 1, 2 ]}\nobreak
\hyperdef{L}{X78EF1A767A0F6FB4}{}
{\noindent\textcolor{FuncColor}{$\triangleright$\enspace\texttt{IsLienEngel({\mdseries\slshape KG})\index{IsLienEngel@\texttt{IsLienEngel}}
\label{IsLienEngel}
}\hfill{\scriptsize (function)}}\\


  For the group ring \texttt{KG} the function \texttt{IsLienEngel} returns \texttt{true} if \texttt{KG} is Lie-n Engel and \texttt{false} otherwise. 
\begin{Verbatim}[commandchars=!@|,fontsize=\small,frame=single,label=Example]
  
  !gapprompt@gap>| !gapinput@G:=CyclicGroup(4);
|
  <pc group of size 4 with 2 generators>
  !gapprompt@gap>| !gapinput@KG:=GroupRing(GF(2),G);
|
  <algebra-with-one over GF(2), with 2 generators>
  !gapprompt@gap>| !gapinput@IsLienEngel(KG);
|
  false
  gap>
  gap>
  !gapprompt@gap>| !gapinput@G:=DihedralGroup(16);
|
  <pc group of size 16 with 4 generators>
  !gapprompt@gap>| !gapinput@KG:=GroupRing(GF(2),G);
|
  <algebra-with-one over GF(2), with 4 generators>
  !gapprompt@gap>| !gapinput@IsLienEngel(KG);
|
  
\end{Verbatim}
 }

 }

 
\section{\textcolor{Chapter }{Random Methods for Obtaining Elements With Desired Properties}}\label{random_methods_elements}
\logpage{[ 2, 2, 0 ]}
\hyperdef{L}{X7A319836781AF22F}{}
{
  

\subsection{\textcolor{Chapter }{GetRandomUnit}}
\logpage{[ 2, 2, 1 ]}\nobreak
\hyperdef{L}{X863101828410DC56}{}
{\noindent\textcolor{FuncColor}{$\triangleright$\enspace\texttt{GetRandomUnit({\mdseries\slshape KG})\index{GetRandomUnit@\texttt{GetRandomUnit}}
\label{GetRandomUnit}
}\hfill{\scriptsize (function)}}\\


  For the group ring \texttt{KG} the function \texttt{GetRandomUnit} returns an unit (i.e. an invertible element) in a random way. 
\begin{Verbatim}[commandchars=!@|,fontsize=\small,frame=single,label=Example]
  
  !gapprompt@gap>| !gapinput@G:=CyclicGroup(4);
|
  <pc group of size 4 with 2 generators>
  !gapprompt@gap>| !gapinput@KG:=GroupRing(GF(7),G);
|
  <algebra-with-one over GF(7), with 2 generators>
  !gapprompt@gap>| !gapinput@u:=GetRandomUnit(KG);;
|
  !gapprompt@gap>| !gapinput@Augmentation(u);
|
  Z(7)^4
  !gapprompt@gap>| !gapinput@u*u^-1;
|
  (Z(7)^0)*<identity> of ...
  
\end{Verbatim}
 }

 

\subsection{\textcolor{Chapter }{GetRandomNormalizedUnit}}
\logpage{[ 2, 2, 2 ]}\nobreak
\hyperdef{L}{X808375C77F186B19}{}
{\noindent\textcolor{FuncColor}{$\triangleright$\enspace\texttt{GetRandomNormalizedUnit({\mdseries\slshape KG})\index{GetRandomNormalizedUnit@\texttt{GetRandomNormalizedUnit}}
\label{GetRandomNormalizedUnit}
}\hfill{\scriptsize (function)}}\\


  For the group ring \texttt{KG} the function \texttt{GetRandomNormalizedUnit} returns a normalized unit (i.e. an invertible element with augmentation 1) in
a random way. 
\begin{Verbatim}[commandchars=!@|,fontsize=\small,frame=single,label=Example]
  
  !gapprompt@gap>| !gapinput@G:=DihedralGroup(IsFpGroup,16);;
|
  !gapprompt@gap>| !gapinput@KG:=GroupRing(GF(2),G);;
|
  !gapprompt@gap>| !gapinput@u:=GetRandomNormalizedUnit(KG);;
|
  !gapprompt@gap>| !gapinput@Augmentation(u);
|
  Z(2)^0
  !gapprompt@gap>| !gapinput@u*u^-1;
|
  (Z(2)^0)*<identity ...>
  
\end{Verbatim}
 }

 

\subsection{\textcolor{Chapter }{GetRandomNormalizedUnitaryUnit}}
\logpage{[ 2, 2, 3 ]}\nobreak
\hyperdef{L}{X7B418FF97BADFFE4}{}
{\noindent\textcolor{FuncColor}{$\triangleright$\enspace\texttt{GetRandomNormalizedUnitaryUnit({\mdseries\slshape KG})\index{GetRandomNormalizedUnitaryUnit@\texttt{GetRandomNormalizedUnitaryUnit}}
\label{GetRandomNormalizedUnitaryUnit}
}\hfill{\scriptsize (function)}}\\


  For the group ring \texttt{KG} the function \texttt{GetRandomNormalizedUnitaryUnit} returns a normalized unitary unit (i.e. such an invertible element with
augmentation 1, that $u\cdot u^{*}=One(KG)$) in a random way. Also, there exists a two-parametrical version of this
method, where the second parameter $\sigma$ is an arbitrary involution. 
\begin{Verbatim}[commandchars=!@|,fontsize=\small,frame=single,label=Example]
  
  !gapprompt@gap>| !gapinput@G:=CyclicGroup(4);;
|
  !gapprompt@gap>| !gapinput@KG:=GroupRing(GF(2),G);;
|
  !gapprompt@gap>| !gapinput@u:=GetRandomNormalizedUnitaryUnit(KG);;
|
  !gapprompt@gap>| !gapinput@u*Involution(u);
|
  (Z(2)^0)*<identity> of ...
  !gapprompt@gap>| !gapinput@Augmentation(u);
|
  Z(2)^0
  
\end{Verbatim}
 }

 

\subsection{\textcolor{Chapter }{GetRandomCentralNormalizedUnit}}
\logpage{[ 2, 2, 4 ]}\nobreak
\hyperdef{L}{X81DF8AB984260200}{}
{\noindent\textcolor{FuncColor}{$\triangleright$\enspace\texttt{GetRandomCentralNormalizedUnit({\mdseries\slshape KG})\index{GetRandomCentralNormalizedUnit@\texttt{GetRandomCentralNormalizedUnit}}
\label{GetRandomCentralNormalizedUnit}
}\hfill{\scriptsize (function)}}\\


  For the group ring \texttt{KG} the function \texttt{GetRandomCentralNormalizedUnit} returns a central normalized unit (i.e. such an invertible element with
augmentation 1, that $u\cdot x=x \cdot u$, $\forall x \in KG$) in a random way. 
\begin{Verbatim}[commandchars=!@|,fontsize=\small,frame=single,label=Example]
  
  !gapprompt@gap>| !gapinput@G:=CyclicGroup(IsFpGroup,4);;
|
  !gapprompt@gap>| !gapinput@KG:=GroupRing(GF(2),G);;
|
  !gapprompt@gap>| !gapinput@u:=GetRandomCentralNormalizedUnit(KG);;
|
  !gapprompt@gap>| !gapinput@Augmentation(u);
|
  Z(2)^0
  !gapprompt@gap>| !gapinput@bool:=true;
|
  true
  !gapprompt@gap>| !gapinput@for x in Elements(KG) do
|
  !gapprompt@>| !gapinput@if x*u<>u*x then bool:=false; break; fi;
|
  !gapprompt@>| !gapinput@od;
|
  !gapprompt@gap>| !gapinput@bool;
|
  true
  
\end{Verbatim}
 }

 

\subsection{\textcolor{Chapter }{GetRandomElementFromAugmentationIdeal}}
\logpage{[ 2, 2, 5 ]}\nobreak
\hyperdef{L}{X78D5FB49877A050E}{}
{\noindent\textcolor{FuncColor}{$\triangleright$\enspace\texttt{GetRandomElementFromAugmentationIdeal({\mdseries\slshape KG})\index{GetRandomElementFromAugmentationIdeal@\texttt{Get}\-\texttt{Random}\-\texttt{Element}\-\texttt{From}\-\texttt{Augmentation}\-\texttt{Ideal}}
\label{GetRandomElementFromAugmentationIdeal}
}\hfill{\scriptsize (function)}}\\


  For the group ring \texttt{KG} the function \texttt{GetRandomElementFromAugmentationIdeal} returns an element from augmentation ideal of $KG$. 
\begin{Verbatim}[commandchars=!@|,fontsize=\small,frame=single,label=Example]
  
  !gapprompt@gap>| !gapinput@G:=QuaternionGroup(16);
|
  <pc group of size 16 with 4 generators>
  !gapprompt@gap>| !gapinput@KG:=GroupRing(GF(2),G);
|
  <algebra-with-one over GF(2), with 4 generators>
  !gapprompt@gap>| !gapinput@u:=GetRandomElementFromAugmentationIdeal(KG);;
|
  !gapprompt@gap>| !gapinput@Augmentation(u);
|
  0*Z(2)
  
\end{Verbatim}
 }

 }

 
\section{\textcolor{Chapter }{Random Methods for Group Rings}}\label{random_methods_general}
\logpage{[ 2, 3, 0 ]}
\hyperdef{L}{X7C53088F8472E3C2}{}
{
  

\subsection{\textcolor{Chapter }{RandomLienEngelLength}}
\logpage{[ 2, 3, 1 ]}\nobreak
\hyperdef{L}{X7FE9D0597F72664F}{}
{\noindent\textcolor{FuncColor}{$\triangleright$\enspace\texttt{RandomLienEngelLength({\mdseries\slshape KG, num})\index{RandomLienEngelLength@\texttt{RandomLienEngelLength}}
\label{RandomLienEngelLength}
}\hfill{\scriptsize (function)}}\\


  Let \texttt{KG} be a group ring and let $[x,y,y,\ldots,y]=0$ for all $x, y \in KG $. Then the number of $y$'s in the last equation is called the Lie n-Engel length. 

 For the group ring \texttt{KG} and the maximal number of iterations \texttt{num} the function \texttt{RandomLienEngelLength} returns the Lie n-Engel length of KG by a random way. 
\begin{Verbatim}[commandchars=!@|,fontsize=\small,frame=single,label=Example]
  
  !gapprompt@gap>| !gapinput@G:=DihedralGroup(16);;
|
  !gapprompt@gap>| !gapinput@KG:=GroupRing(GF(2),G);;
|
  !gapprompt@gap>| !gapinput@RandomLienEngelLength(KG,100);
|
  4
  
\end{Verbatim}
 }

 

\subsection{\textcolor{Chapter }{RandomExponent}}
\logpage{[ 2, 3, 2 ]}\nobreak
\hyperdef{L}{X78B198507C9F53CE}{}
{\noindent\textcolor{FuncColor}{$\triangleright$\enspace\texttt{RandomExponent({\mdseries\slshape KG, num})\index{RandomExponent@\texttt{RandomExponent}}
\label{RandomExponent}
}\hfill{\scriptsize (function)}}\\


  For the group ring \texttt{KG} and the maximal number of iterations \texttt{num} the function \texttt{RandomExponent} returns the exponent of the group of normalized units of \texttt{KG} by a random way. 
\begin{Verbatim}[commandchars=!@|,fontsize=\small,frame=single,label=Example]
  
  !gapprompt@gap>| !gapinput@G:=DihedralGroup(16);;
|
  !gapprompt@gap>| !gapinput@KG:=GroupRing(GF(2),G);;
|
  !gapprompt@gap>| !gapinput@RandomExponent(KG,100);
|
  8
  
\end{Verbatim}
 }

 

\subsection{\textcolor{Chapter }{RandomExponentOfNormalizedUnitsCenter}}
\logpage{[ 2, 3, 3 ]}\nobreak
\hyperdef{L}{X81106C3E808ABAB4}{}
{\noindent\textcolor{FuncColor}{$\triangleright$\enspace\texttt{RandomExponentOfNormalizedUnitsCenter({\mdseries\slshape KG, num})\index{RandomExponentOfNormalizedUnitsCenter@\texttt{Random}\-\texttt{Exponent}\-\texttt{Of}\-\texttt{Normalized}\-\texttt{Units}\-\texttt{Center}}
\label{RandomExponentOfNormalizedUnitsCenter}
}\hfill{\scriptsize (function)}}\\


  For the group ring \texttt{KG} and the maximal number of iterations \texttt{num} the function \texttt{RandomExponentOfNormalizedUnitsCenter} returns the exponent of the center of the group of normalized units of \texttt{KG} by a random way. 
\begin{Verbatim}[commandchars=!@|,fontsize=\small,frame=single,label=Example]
  
  !gapprompt@gap>| !gapinput@G:=DihedralGroup(16);;
|
  !gapprompt@gap>| !gapinput@KG:=GroupRing(GF(2),G);;
|
  !gapprompt@gap>| !gapinput@RandomExponentOfNormalizedUnitsCenter(KG,100);
|
  4
  
\end{Verbatim}
 }

 

\subsection{\textcolor{Chapter }{RandomNilpotencyClass}}
\logpage{[ 2, 3, 4 ]}\nobreak
\hyperdef{L}{X7D4AB0007DD1F690}{}
{\noindent\textcolor{FuncColor}{$\triangleright$\enspace\texttt{RandomNilpotencyClass({\mdseries\slshape KG, num})\index{RandomNilpotencyClass@\texttt{RandomNilpotencyClass}}
\label{RandomNilpotencyClass}
}\hfill{\scriptsize (function)}}\\


  For the group ring \texttt{KG} and the maximal number of iterations \texttt{num} the function \texttt{RandomNilpotencyClass} returns the nilpotency class of the group of normalized units of \texttt{KG} by a random way. 
\begin{Verbatim}[commandchars=!@|,fontsize=\small,frame=single,label=Example]
  
  !gapprompt@gap>| !gapinput@G:=DihedralGroup(16);;
|
  !gapprompt@gap>| !gapinput@KG:=GroupRing(GF(2),G);;
|
  !gapprompt@gap>| !gapinput@RandomNilpotencyClass(KG,100);
|
  4
  
\end{Verbatim}
 }

 

\subsection{\textcolor{Chapter }{RandomDerivedLength}}
\logpage{[ 2, 3, 5 ]}\nobreak
\hyperdef{L}{X8177059281A19CB3}{}
{\noindent\textcolor{FuncColor}{$\triangleright$\enspace\texttt{RandomDerivedLength({\mdseries\slshape KG, n})\index{RandomDerivedLength@\texttt{RandomDerivedLength}}
\label{RandomDerivedLength}
}\hfill{\scriptsize (function)}}\\


 $FG$ is called \mbox{\texttt{\mdseries\slshape Lie solvable}}, if some of the terms of the Lie derived series $\delta^{[n]}(FG)=[\delta^{[n-1]}(FG),\delta^{[n-1]}(FG)]$ with $\delta^{[0]}(FG)=FG$ are equal to zero. 

 Denote by $\dl_L(FG)$ the minimal element of the set $\{m\in\mathbb N\;\vert\; \delta^{[m]}(FG)=0\}$, which is said to be the \mbox{\texttt{\mdseries\slshape Lie derived length}} of $FG$. 

 For the group ring \texttt{KG} and a positive integer \texttt{n} the function \texttt{RandomDerivedLength} returns the Lie derived length by a random way. 
\begin{Verbatim}[commandchars=!@|,fontsize=\small,frame=single,label=Example]
  
  !gapprompt@gap>| !gapinput@D:=DihedralGroup(IsFpGroup,8);;
|
  !gapprompt@gap>| !gapinput@KG:=GroupRing(GF(2),D);;
|
  !gapprompt@gap>| !gapinput@RandomDerivedLength(KG,100);
|
  2
  
\end{Verbatim}
 }

 

\subsection{\textcolor{Chapter }{RandomCommutatorSubgroup}}
\logpage{[ 2, 3, 6 ]}\nobreak
\hyperdef{L}{X8080A12D78AC57F9}{}
{\noindent\textcolor{FuncColor}{$\triangleright$\enspace\texttt{RandomCommutatorSubgroup({\mdseries\slshape KG, n})\index{RandomCommutatorSubgroup@\texttt{RandomCommutatorSubgroup}}
\label{RandomCommutatorSubgroup}
}\hfill{\scriptsize (function)}}\\


 For the group ring \texttt{KG} and a positive integer \texttt{n} the function \texttt{RandomCommutatorSubgroup} returns the commutator subgroup by a random way. 
\begin{Verbatim}[commandchars=!@|,fontsize=\small,frame=single,label=Example]
  
  !gapprompt@gap>| !gapinput@G:=DihedralGroup(IsFpGroup,8);;
|
  !gapprompt@gap>| !gapinput@KG:=GroupRing(GF(2),G);;
|
  !gapprompt@gap>| !gapinput@SG:=RandomCommutatorSubgroup(KG,100);
|
  !gapprompt@gap>| !gapinput@StructureDescription(SG);
|
  "C2 x C2 x C2"
  !gapprompt@gap>| !gapinput@G:=CyclicGroup(8);;
|
  !gapprompt@gap>| !gapinput@KG:=GroupRing(GF(3),G);;
|
  !gapprompt@gap>| !gapinput@SG:=RandomCommutatorSubgroup(KG,100);;
|
  !gapprompt@gap>| !gapinput@Elements(SG);
|
  [ (Z(3)^0)*<identity> of ... ]
  
\end{Verbatim}
 }

 

\subsection{\textcolor{Chapter }{RandomCommutatorSubgroupOfNormalizedUnits}}
\logpage{[ 2, 3, 7 ]}\nobreak
\hyperdef{L}{X7E3BA1487A44E09D}{}
{\noindent\textcolor{FuncColor}{$\triangleright$\enspace\texttt{RandomCommutatorSubgroupOfNormalizedUnits({\mdseries\slshape KG, n})\index{RandomCommutatorSubgroupOfNormalizedUnits@\texttt{Random}\-\texttt{Commutator}\-\texttt{Subgroup}\-\texttt{Of}\-\texttt{Normalized}\-\texttt{Units}}
\label{RandomCommutatorSubgroupOfNormalizedUnits}
}\hfill{\scriptsize (function)}}\\


 For the group ring \texttt{KG} and a positive integer \texttt{n} the function \texttt{RandomCommutatorSubgroupOfNormalizedUnits} returns the commutator subgroup of normalized units by a random way. 
\begin{Verbatim}[commandchars=!@|,fontsize=\small,frame=single,label=Example]
  
  !gapprompt@gap>| !gapinput@G:=DihedralGroup(8);;
|
  !gapprompt@gap>| !gapinput@KG:=GroupRing(GF(3),G);;
|
  !gapprompt@gap>| !gapinput@SG:=RandomCommutatorSubgroup(KG,100);;
|
  !gapprompt@gap>| !gapinput@u:=Random(Elements(SG));;
|
  !gapprompt@gap>| !gapinput@Augmentation(u);
|
  Z(3)^0
  
\end{Verbatim}
 }

 

\subsection{\textcolor{Chapter }{RandomCenterOfCommutatorSubgroup}}
\logpage{[ 2, 3, 8 ]}\nobreak
\hyperdef{L}{X8672D3997E1C4EFF}{}
{\noindent\textcolor{FuncColor}{$\triangleright$\enspace\texttt{RandomCenterOfCommutatorSubgroup({\mdseries\slshape KG, n})\index{RandomCenterOfCommutatorSubgroup@\texttt{RandomCenterOfCommutatorSubgroup}}
\label{RandomCenterOfCommutatorSubgroup}
}\hfill{\scriptsize (function)}}\\


 For the group ring \texttt{KG} and a positive integer \texttt{n} the function \texttt{RandomCenterOfCommutatorSubgroup} returns the center of the commutator subgroup by a random way. 
\begin{Verbatim}[commandchars=!@|,fontsize=\small,frame=single,label=Example]
  
  !gapprompt@gap>| !gapinput@G:=DihedralGroup(8);;
|
  !gapprompt@gap>| !gapinput@KG:=GroupRing(GF(3),G);;
|
  !gapprompt@gap>| !gapinput@SG:=RandomCenterOfCommutatorSubgroup(KG,100);;
|
  !gapprompt@gap>| !gapinput@x1:=Random(Elements(SG));; x2:=Random(Elements(SG));;
|
  !gapprompt@gap>| !gapinput@x1*x2=x2*x1;
|
  true
  
\end{Verbatim}
 }

 

\subsection{\textcolor{Chapter }{RandomNormalizedUnitGroup}}
\logpage{[ 2, 3, 9 ]}\nobreak
\hyperdef{L}{X81D74AB97C8370B8}{}
{\noindent\textcolor{FuncColor}{$\triangleright$\enspace\texttt{RandomNormalizedUnitGroup({\mdseries\slshape KG, n})\index{RandomNormalizedUnitGroup@\texttt{RandomNormalizedUnitGroup}}
\label{RandomNormalizedUnitGroup}
}\hfill{\scriptsize (function)}}\\


 For the group ring \texttt{KG} and a positive integer \texttt{n} the function \texttt{RandomNormalizedUnitGroup} returns the normalized unit group by a random way. 
\begin{Verbatim}[commandchars=!@|,fontsize=\small,frame=single,label=Example]
  
  !gapprompt@gap>| !gapinput@G:=DihedralGroup(8);;
|
  !gapprompt@gap>| !gapinput@KG:=GroupRing(GF(2),G);;
|
  !gapprompt@gap>| !gapinput@SG:=RandomNormalizedUnitGroup(KG);
|
  <group with 4 generators>
  !gapprompt@gap>| !gapinput@Size(SG);
|
  128
  !gapprompt@gap>| !gapinput@u:=Random(Elements(SG));;
|
  !gapprompt@gap>| !gapinput@Augmentation(u);
|
  Z(2)^0
  
\end{Verbatim}
 }

 

\subsection{\textcolor{Chapter }{RandomUnitarySubgroup}}
\logpage{[ 2, 3, 10 ]}\nobreak
\hyperdef{L}{X82A6FD3384663AE8}{}
{\noindent\textcolor{FuncColor}{$\triangleright$\enspace\texttt{RandomUnitarySubgroup({\mdseries\slshape KG, n})\index{RandomUnitarySubgroup@\texttt{RandomUnitarySubgroup}}
\label{RandomUnitarySubgroup}
}\hfill{\scriptsize (function)}}\\


 For the group ring \texttt{KG} and a positive integer \texttt{n} the function \texttt{RandomUnitarySubgroup} returns the unitary subgroup by a random way. 
\begin{Verbatim}[commandchars=!@|,fontsize=\small,frame=single,label=Example]
  
  !gapprompt@gap>| !gapinput@G:=DihedralGroup(8);;
|
  !gapprompt@gap>| !gapinput@KG:=GroupRing(GF(3),G);;
|
  !gapprompt@gap>| !gapinput@SG:=RandomUnitarySubgroup(KG,100);;
|
  !gapprompt@gap>| !gapinput@u:=Random(Elements(SG));;
|
  !gapprompt@gap>| !gapinput@Augmentation(u);
|
  Z(3)^0
  !gapprompt@gap>| !gapinput@u*u^-1;
|
  (Z(3)^0)*<identity> of ...
  
\end{Verbatim}
 }

 

\subsection{\textcolor{Chapter }{RandomCommutatorSeries}}
\logpage{[ 2, 3, 11 ]}\nobreak
\hyperdef{L}{X7D95FC4287C52C30}{}
{\noindent\textcolor{FuncColor}{$\triangleright$\enspace\texttt{RandomCommutatorSeries({\mdseries\slshape KG, n})\index{RandomCommutatorSeries@\texttt{RandomCommutatorSeries}}
\label{RandomCommutatorSeries}
}\hfill{\scriptsize (function)}}\\


 For the group ring \texttt{KG} and a positive integer \texttt{n} the function \texttt{RandomCommutatorSeries} returns the commutator series by a random way. 
\begin{Verbatim}[commandchars=!@|,fontsize=\small,frame=single,label=Example]
  
  !gapprompt@gap>| !gapinput@G:=DihedralGroup(8);;
|
  !gapprompt@gap>| !gapinput@KG:=GroupRing(GF(2),G);;
|
  !gapprompt@gap>| !gapinput@CS:=RandomCommutatorSeries(KG,100);
|
  [ <group of size 128 with 4 generators>, <group of size 8 with 8 generators>,
    <group of size 1 with 1 generators> ]
  
\end{Verbatim}
 }

 

\subsection{\textcolor{Chapter }{RandomLowerCentralSeries}}
\logpage{[ 2, 3, 12 ]}\nobreak
\hyperdef{L}{X7B6147BA7D99BD63}{}
{\noindent\textcolor{FuncColor}{$\triangleright$\enspace\texttt{RandomLowerCentralSeries({\mdseries\slshape KG, n})\index{RandomLowerCentralSeries@\texttt{RandomLowerCentralSeries}}
\label{RandomLowerCentralSeries}
}\hfill{\scriptsize (function)}}\\


 For the group ring \texttt{KG} and a positive integer \texttt{n} the function \texttt{RandomLowerCentralSeries} returns the lower central series by a random way. 
\begin{Verbatim}[commandchars=!@|,fontsize=\small,frame=single,label=Example]
  
  !gapprompt@gap>| !gapinput@G:=DihedralGroup(8);;
|
  !gapprompt@gap>| !gapinput@KG:=GroupRing(GF(2),G);;
|
  !gapprompt@gap>| !gapinput@CS:=RandomLowerCentralSeries(KG,100);
|
  [ <group of size 128 with 4 generators>, <group of size 8 with 8 generators>,
    <group of size 1 with 1 generators> ]
  
\end{Verbatim}
 }

 

\subsection{\textcolor{Chapter }{RandomUnitaryOrder}}
\logpage{[ 2, 3, 13 ]}\nobreak
\hyperdef{L}{X7C65940B7A332CCA}{}
{\noindent\textcolor{FuncColor}{$\triangleright$\enspace\texttt{RandomUnitaryOrder({\mdseries\slshape KG, n})\index{RandomUnitaryOrder@\texttt{RandomUnitaryOrder}}
\label{RandomUnitaryOrder}
}\hfill{\scriptsize (function)}}\\


 For the group ring \texttt{KG} and a positive integer \texttt{n} the function \texttt{RandomUnitaryOrder} returns the unitary order of $KG$ by a random way. Also, there exists a three-parametrical version of this
method, where the third parameter $\sigma$ is an arbitrary involution. 
\begin{Verbatim}[commandchars=!@|,fontsize=\small,frame=single,label=Example]
  
  !gapprompt@gap>| !gapinput@G:=DihedralGroup(8);;
|
  !gapprompt@gap>| !gapinput@KG:=GroupRing(GF(3),G);;
|
  !gapprompt@gap>| !gapinput@ord:=RandomUnitaryOrder(KG,100);
|
  243
  
\end{Verbatim}
 }

 

\subsection{\textcolor{Chapter }{RandomDihedralDepth}}
\logpage{[ 2, 3, 14 ]}\nobreak
\hyperdef{L}{X811670E4829B2291}{}
{\noindent\textcolor{FuncColor}{$\triangleright$\enspace\texttt{RandomDihedralDepth({\mdseries\slshape KG, n})\index{RandomDihedralDepth@\texttt{RandomDihedralDepth}}
\label{RandomDihedralDepth}
}\hfill{\scriptsize (function)}}\\


 For the group ring \texttt{KG} and a positive integer \texttt{n} the function \texttt{RandomCentralUnitaryOrder} returns the depth of $KG$ by a random way. 
\begin{Verbatim}[commandchars=!@|,fontsize=\small,frame=single,label=Example]
  
  !gapprompt@gap>| !gapinput@G:=DihedralGroup(16);;
|
  !gapprompt@gap>| !gapinput@UD:=PcNormalizedUnitGroup(KG);
|
  <pc group of size 32768 with 15 generators>
  !gapprompt@gap>| !gapinput@DihedralDepth(UD);
|
  3
  !gapprompt@gap>| !gapinput@time;
|
  4211
  !gapprompt@gap>| !gapinput@RandomDihedralDepth(UD,100);
|
  2
  !gapprompt@gap>| !gapinput@RandomDihedralDepth(UD,300);
|
  0
  !gapprompt@gap>| !gapinput@RandomDihedralDepth(UD,500);
|
  2
  !gapprompt@gap>| !gapinput@RandomDihedralDepth(UD,1000);
|
  0
  !gapprompt@gap>| !gapinput@RandomDihedralDepth(UD,1000);
|
  
\end{Verbatim}
 }

 }

 }

  \def\bibname{References\logpage{[ "Bib", 0, 0 ]}
\hyperdef{L}{X7A6F98FD85F02BFE}{}
}

\bibliographystyle{alpha}
\bibliography{ramegabib.xml}

\addcontentsline{toc}{chapter}{References}

\def\indexname{Index\logpage{[ "Ind", 0, 0 ]}
\hyperdef{L}{X83A0356F839C696F}{}
}

\cleardoublepage
\phantomsection
\addcontentsline{toc}{chapter}{Index}


\printindex

\newpage
\immediate\write\pagenrlog{["End"], \arabic{page}];}
\immediate\closeout\pagenrlog
\end{document}
